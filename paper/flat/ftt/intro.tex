\section{An Overview of Flat Type Theory}

Before looking at \this, it is important to understand Hierarchal Type Theory as well as Prolog's answer set theory.
To start, let us look at how we repersent the statement "for any natural numbers $A$ and $B$, if $A$ is greater then or equal to $B$, then the succesor to $A$ must be greater then or equal to the successor for $B$".
We might write this is a simple logic as $\forall (A : \nat) \forall (B : \nat) (A = B \proves S(A) = S(B))$.

When representing this in Hierarchal Type Theory\footnote{With the name "sgeq"}, we might write it as $\seq :: \fdt{A : \nat} \fdt{B : \nat} A \geq B \to S(A) \geq S(B)$, where $\Pi$ denotates dependent function types.
The notion behind this is that if there exists a mapping (namely sgeq) from the statement $A \geq B$ to $S(A) \geq S(B)$, thence we know that if we have a given statement of the form $A \geq B$, because $\seq$ is a complete mapping from $A \geq B$ to $S(A) \geq S(B)$, we will get a valid argument.

However in Prolog, we would instead write it like this: \verb|S(A) >= S(B) :- A >= B, nat(B), nat(A).|
While at first glance these might seem disimilar, if I write the set theoretic equivalent to the type-theoretical version, using $\phi_S$ to mean "the proposition that determines if something is in $S$", then we could instead write it as $\forall A \forall B (S(A) \geq S(B) \leftarrow A \geq B \leftarrow \phi_\nat (B) \leftarrow \phi_\nat (A)) $)

\begin{math}
	\mathtt{app} :: \pdt{f :: A \to B} \fdt{a :: A} B \\
	(::) :: \fdt{a::A} A
\end{math}