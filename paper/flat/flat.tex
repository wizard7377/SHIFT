\part{Flat Type Theory}

A quick note before I delve into Flat Type Theory.
Here, I will reference "metavariables", "metalogic", and certain symbols.
These are statements that underline how the system works, and are written in natural language or symbolically.

All symbolic metalogic statements should just be understood as a shorthand for the given statement in natural language, and neither can be formalized.

Because I will use certain symbols for convience, I shall note them all here in advance.

The lowercase (minuscule) Greek Letters, $\alpha$...$\omega$, excluding $\phi$ (and for reasons of simplicty $\lambda$ is also not used), repersent "metavariables", that is, any sequence of symbols that acts as a single value.
Each of these metavariables is scoped over the expression, so when, say $\alpha$ is used in the statements $\alpha : \yud$ and $\alpha : \alpha$, these refer to different $\alpha$s.
Each instance of these must be the same as any other occurence in scope, but may be the same as some over metavariable.

That is to say, the metavariables are bindings from lowercase Greek letters to sequences of symbols, with that binding being pure but not neccaisarlly unique.

As mentioned before, the letter $\phi$ is special.
It denotates a given $\lambda$ on a given term to a given expression, that is, it is a given bounded expression.
For instance, we might have $\phi(\_) \equiv \_ + \_$, where for instance $\phi(1) = 1 + 1$.
In the case mutiple are needed, they shall be differentiated by subscript, $\phi_0$, $\phi_1$ etc..

The operator $\mand$ shall be read as "and", that is, if in some sense the statement to the left is true, and in the same sense the statement to the right is true, then the whole statement is true.

Finally, $\proves$ shall be read as "proves", that is if whatever to the left in some sense true, then whatever is to the right must also be in that same sense true.
For utility, $\dashv$ is defined as "is proven by", such that $\alpha \proves \beta$ is equivalent to $\beta \dashv \alpha$, and $\alpha \bip \beta$ expands to the statements $\alpha \proves \beta$ and $\beta \proves \alpha$ (this is meant to just be the symbols $\proves$ and $\dashv$ overlaid on top of each other)

\todo{This is mostly fine, just needs to be a little longer}

%\section{Basics of Flat Types from the Perspective Of Classical Type Theory}

Most type theories, including Typed $\lambda$-calculus, Intuitionistic Type Theory, Homotopy Type Theory, and the Calculus of Constructions, all to some extent try to formalize the notion of a proof \needcite.
However, what very \emph{few} of them try to do is formalize a basic system for \emph{logic} \needcite.

In addition, type theories are often not very strongly based on set theory, which, for attempting to be a system for types, or collections of things, seems to be a large change \needcite.
Rather than focusing on (as set theory does) the notion of $A$ is \emph{in} $B$, it instead focuses on statement such as $A {=}_{B} C$ and $A \to B$, which, while interesting and useful, often only bares a loose resemblance to actually collections of elements

\subsection{Logic, Sets, and Types}

Set theory, as a subject, was very much based on mathmatical logic.
Indeed, the axiom of specification is the statement that for some proposition, $P$, and some set, $S$, it is possible to construct all the elements of $S$ that satisfy $P$ \needcite.

While this is interesting in even simple cases, for instance, over the natural numbers asking which of them are some other natural number divided by 2 (the evens), it gets more interesting if we start asking about more complex examples.

For instance, for a given set, what set describes the proposition for 2 terms of that set, "the first term is greater than or equal to the second"?
Unlike the even numbers, this is much less clear.

To get our answer, let us instead consider a more specific case of instead of any set, the natural numbers, and instead of a proposition of two terms instead we will ask about "this term is greater than or equal to 2". 
In other words, we have a proposition of arity one, and that returns whether a given natural number is greater than 2.

This then is much more clear, where we simply have the set $\{2,3,4,...\}$ as our result.
So, for the natural numbers, what is the general case?
We know that for any given natural number $m$, if we partial apply the above proposition we get $\lambda n . (n \geq m)$.
So, now for those familiar with functional programming, the idea of transforming the above term of sets into a set of pairs should make much sense, and we get a list of natural numbers where the first is greater than the second.

\subsection{Impossible Falsehood}

As discussed in \ref{paradox}, there don't exist the usual combination of (at least) three states of sentences, those being provably true, unprovable, and provably first.
The rationale for this was already mentioned, so I won't say it again.

However, I will talk about what exactly \this \emph{does} have.
Instead of those statuses mentioned above, a statement may either be constructed or unconstructed.
A constructed statement is a well formed, provably true statement, that is, statements that are known to be true.
A unconstructed statement is anything else, including (syntactically) ill-formed statements, (semantically) false statements, and statements that can not or have not been proven to be construcable.

This means that it is \emph{impossible} to prove a false statement.
In terms of formal computability, a program that tries to prove a false statement never terminates.
However, no distinction is drawn between unprovable statements (that are true or false) and false statements.

This means that a statement can never "become" more restrcitive, that is, if you made a logical proof of some statement, it will always be valid, as no-one can make a constructible statement unconstructible.
This, alone, is quite important.
Proofs are always reliable, you can depend on a statement proven true to be true (given your axioms).

In addition, this allows us to further extend the cumulative type theory notion of accumulating universes to simply one universe, where $\yud : \yud$, which is normally not allowed.
This also allows for much nicer proofs, as it is no longer neccasarry to verify that a statement cannot be proven false when trying to prove it true.
%\section{Flat Type Theory from the Perspective of Prolog}

The Prolog programming language is a language that can be considered the "low level" of mathematically founded programming languages.
In the same way that a language like C is low level because it interacts closely with the hardware, a language like Prolog is low level because it works with closely with fundamental theorems.

While Flat Type Theory is a type theory, it actually bears more resemblence to Prolog in this way.
It focuses less on higher level concepts such as classes, and focuses more on lower level concepts, such as implication

The main difference between Prolog and Flat Type Theory is that Flat Type Theory uses higher order logic by default.
That is to say, a clause may always be used as a term and vice versa.
In addition, instead of a depth first search, \this uses a breadth first search, that is, it reasons one level at a time.

These two changes fix a number of problems in standard Prolog. 
In particular, the second one fixes Prolog's problem of having the order that statements are given as a core part of the language, and removing the need for the dreaded \verb|!| (cut) operator. \cite{swipl}
The first change is just a matter of modernizing the Prolog syntax. 
Indeed, second order predicates are \emph{already} included in most Prolog implementations, simply using functor syntax \cite{flach2022simply}.

The first part of this we will go into more in \ref{mecha}.
The second part of this is that \verb|:| (\this 's equivalent to \verb|:-|) can be used to construct terms in addition to clauses.

So, for instance, while the Prolog term \verb|(Cold(Person,Day) :- Wet(Day)) :- Winter(Day)| is invalid, the Flat equivalent is a perfectly valid term \emph{and} clause.
%\section{General Curry-Howard Isomorphism}

This might all of seemed so far to have been a jumble of 
%\section{Formalization of Flat Type Theory}


%\section{Covariance, Invariance, and Contravariance}

One important thing to note is that certain types do \emph{vary} alongside their elements.
For instance, if we have the expression, $\nat + \mathbb{Z}$, we would like to say purely because it is both an expression that contains $\mathbb{Z}$ and the fact that $\nat : \mathbb{Z}$ to say that $\nat + \nat$ is a subtype of $\nat + \mathbb{Z}$.

While in this specific instance this fact happens to hold true, this generalized inference about subterms is not allowed, that is, we would need a specific rule $\lamed A \lamed B \lamed C (A : C \tto (A + B) : (C + B))$ in our assumptions.

Why? 

To answer that, let us take a look at functions.
If we think from the perspective of a programmer, then the statement \verb|func : Fn<A> -> B| \footnote{This is in Rust} is something like a statement, where the function is "saying" something along the lines of 'If you give me something of type $A$, I'll give you something of type $B$ back'.

But, if we consider from the functions perspective, it dosen't \emph{need} to only be defined for this.
It could, for example, just be lazy \footnote{In the "personality" (yes the personality of a function) sense, not in the computational sense} and actually be able to take in, for instance, if $A$ is $\nat$ perhaps it can take any $\mathbb{Z}$ and simply chooses to say that it can only take $\nat$.

In a sense, we might say that we could interpret this in two ways \footnote{For anyone aware of the formalities of this, I know this reasoning isn't exactly strictly formal, but I thought it was fun}, that either \verb|func| is a function limited to $\nat$ or it is just being lazy. 
In a sense, we know less about \verb|func| from \verb|func : Fn<Nat> -> T| then from \verb|func : Fn<Int> -> T| \footnote{No, Rust's standard library doesn't provide Nat or Int However, because of how signedness bits work this example would require types that were one bit apart in size, which can't occur (as almost all languages, including Rust, are byte aligned)}.
\todo{Verb text in foot}

But, if you'll notice, this means that $\sint \to T$ is a subtype of $\nat \to T$, even though $\nat : \sint$.
The reason for this is simple, when a function type has a larger input, it is making more promises about valid uses of the function, and therefore, because there are \emph{less} things that promise more, it is seen as a subtype.

So how does this get back to the addition problem? 
Because, if we could replace any instance of a less specific type with a more specific one and get a more specific type, it would stand to reason that $\nat \to T$ is a subtype of $\sint \to T$, which as we've seen is not true.

Specifacally, over it's input, function types input what is known as contravariance, that is, the more specific the input, the less specific the function type and vice versa.
However, just like both parts of the addition type from before, the output type of a function exhibits covariance, that is, the more speific the subterm the more specific the whole term and vice versa. \cite{Rustonomicon}

Something that occurs much more in programming and much less in pure type theory is what's known as invariance, that is, where the type of a term does not depend on a given subterm   

\section{An Overview of Flat Type Theory}

Before looking at \this, it is important to understand Hierarchal Type Theory as well as Prolog's answer set theory.
To start, let us look at how we repersent the statement "for any natural numbers $A$ and $B$, if $A$ is greater then or equal to $B$, then the succesor to $A$ must be greater then or equal to the successor for $B$".
We might write this is a simple logic as $\forall (A : \nat) \forall (B : \nat) (A = B \proves S(A) = S(B))$.

When representing this in Hierarchal Type Theory\footnote{With the name "sgeq"}, we might write it as $\seq :: \fdt{A : \nat} \fdt{B : \nat} A \geq B \to S(A) \geq S(B)$, where $\Pi$ denotates dependent function types.
The notion behind this is that if there exists a mapping (namely sgeq) from the statement $A \geq B$ to $S(A) \geq S(B)$, thence we know that if we have a given statement of the form $A \geq B$, because $\seq$ is a complete mapping from $A \geq B$ to $S(A) \geq S(B)$, we will get a valid argument.

However in Prolog, we would instead write it like this: \verb|S(A) >= S(B) :- A >= B, nat(B), nat(A).|
While at first glance these might seem disimilar, if I write the set theoretic equivalent to the type-theoretical version, using $\phi_S$ to mean "the proposition that determines if something is in $S$", then we could instead write it as $\forall A \forall B (S(A) \geq S(B) \leftarrow A \geq B \leftarrow \phi_\nat (B) \leftarrow \phi_\nat (A)) $)

\begin{math}
	\mathtt{app} :: \pdt{f :: A \to B} \fdt{a :: A} B \\
	(::) :: \fdt{a::A} A
\end{math}
\section{Reasoning in Flat Type Theory}
\section{Formalization of Flat Type Theory}



\section{Evaluation and Side Effects}
One very important part of this is the relationship between evaluation and subtyping.
That is, we have established that $1+1$ is a type, but what does it contain?

It contains anything that it evaluates to!
For instance, $1+1$ evaluates to $2$, therefore $2 : 1+1$

Why is this useful?
Consider the square root function, $\sqrt{x}$. We would \emph{like} to say that $\sqrt{4} = 2$, but this is not true, as $\sqrt{4}$ is also $-2$.
We might say that the square root of $4$ has two possible interpreations, that of $2$ and $-2$.
And just as we say that because the collection of things refered to by the natural numbers is more specific than the integers, we say that $2 : \sqrt{4}$. 

\subsection{Function Types}
We've already seen function types, but what do these actually mean?
That is, if we have a function $f : A \to B$, what can be said about this?

Well, it is rather simple, if $f : A \to B$, then for any $x$ is a subtype of $A$, then $f(x) : B$.

One very interesting thing about functions are there variance, that is, what the generality of $A$ and $B$ have on $A \to B$.
While many things are either covariant and contravariant, very few are both.
However, the function is, it is contravariant over $A$ and covariant over $B$, that is $(C : A) \Rightarrow (A \to B) : (C \to B)$ and $(B : C) \Rightarrow (A \to B) : (A \to C)$ 

The usage of $\Rightarrow$ here is entierly for the purpose of ease of reading. 
It is just a notation of $A : B$ (from $B \Rightarrow A$), so that these are more readable.
Another notation is $A[x]$, which is equivalent to $x : A$