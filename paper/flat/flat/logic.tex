\section{Basics of Flat Types from the Perspective Of Classical Type Theory}

Most type theories, including Typed $\lambda$-calculus, Intuitionistic Type Theory, Homotopy Type Theory, and the Calculus of Constructions, all to some extent try to formalize the notion of a proof \needcite.
However, what very \emph{few} of them try to do is formalize a basic system for \emph{logic} \needcite.

In addition, type theories are often not very strongly based on set theory, which, for attempting to be a system for types, or collections of things, seems to be a large change \needcite.
Rather than focusing on (as set theory does) the notion of $A$ is \emph{in} $B$, it instead focuses on statement such as $A {=}_{B} C$ and $A \to B$, which, while interesting and useful, often only bares a loose resemblance to actually collections of elements

\subsection{Logic, Sets, and Types}

Set theory, as a subject, was very much based on mathmatical logic.
Indeed, the axiom of specification is the statement that for some proposition, $P$, and some set, $S$, it is possible to construct all the elements of $S$ that satisfy $P$ \needcite.

While this is interesting in even simple cases, for instance, over the natural numbers asking which of them are some other natural number divided by 2 (the evens), it gets more interesting if we start asking about more complex examples.

For instance, for a given set, what set describes the proposition for 2 terms of that set, "the first term is greater than or equal to the second"?
Unlike the even numbers, this is much less clear.

To get our answer, let us instead consider a more specific case of instead of any set, the natural numbers, and instead of a proposition of two terms instead we will ask about "this term is greater than or equal to 2". 
In other words, we have a proposition of arity one, and that returns whether a given natural number is greater than 2.

This then is much more clear, where we simply have the set $\{2,3,4,...\}$ as our result.
So, for the natural numbers, what is the general case?
We know that for any given natural number $m$, if we partial apply the above proposition we get $\lambda n . (n \geq m)$.
So, now for those familiar with functional programming, the idea of transforming the above term of sets into a set of pairs should make much sense, and we get a list of natural numbers where the first is greater than the second.

\subsection{Impossible Falsehood}

As discussed in \ref{paradox}, there don't exist the usual combination of (at least) three states of sentences, those being provably true, unprovable, and provably first.
The rationale for this was already mentioned, so I won't say it again.

However, I will talk about what exactly \this \emph{does} have.
Instead of those statuses mentioned above, a statement may either be constructed or unconstructed.
A constructed statement is a well formed, provably true statement, that is, statements that are known to be true.
A unconstructed statement is anything else, including (syntactically) ill-formed statements, (semantically) false statements, and statements that can not or have not been proven to be construcable.

This means that it is \emph{impossible} to prove a false statement.
In terms of formal computability, a program that tries to prove a false statement never terminates.
However, no distinction is drawn between unprovable statements (that are true or false) and false statements.

This means that a statement can never "become" more restrcitive, that is, if you made a logical proof of some statement, it will always be valid, as no-one can make a constructible statement unconstructible.
This, alone, is quite important.
Proofs are always reliable, you can depend on a statement proven true to be true (given your axioms).

In addition, this allows us to further extend the cumulative type theory notion of accumulating universes to simply one universe, where $\yud : \yud$, which is normally not allowed.
This also allows for much nicer proofs, as it is no longer neccasarry to verify that a statement cannot be proven false when trying to prove it true.